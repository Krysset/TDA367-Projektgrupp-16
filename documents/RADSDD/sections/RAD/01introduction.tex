\section{Introduction}

\begin{comment}
Give some background and explain the purpose of this application. Describe
the functionality of the application. Describe the stakeholders of the project,
highlight who will benefit from/use this particular application.
\end{comment}
\subsection{The Background}
Feyrune, as is the name of the application, was first conceptualized in the beginning as a Pokémon-style RPG, but with the ability to be easily extended with new functionality to differentiate itself. The purpose of Feyrune is to bring joy to whomever chooses to use it.

\subsection{Functionality}
Functionality wise the game has two main areas, the overworld and combat. The overworld pertains to functionality such as moving the player around the map, and travelling to new areas. Thanks to the map parser, adding new maps is as easy as creating a map in Tiled\cite{tiled} and adding it to the project.\\
\\
Combat is the other main area of functionality, and is where the player will be able to fight against enemies. The combat system is turn-based, and will be able to be extended with new functionality to make it more interesting. These two different areas of functionality is glued together by an encounter system, which determines when the player encounters an enemy and initiates combat with them.

\subsection{Stakeholders}
The stakeholders for the project is the general populace, but mostly the people who are interested in RPGs and games with tactical The game will be free to play and available on Windows, Linux and Mac OS X. The game will be free to play and will be available on PC.


\subsection{Definitions, acronyms, and abbreviations}
\begin{comment}
Create a word list to avoid confusion and give a definition of every abbreviation
you use in the document.
\end{comment}
\label{wordlist}
\begin{tabbing}
	\indent\= Feyrune:\indent\indent\= The name of the application\\
	\> Hej\> hej
\end{tabbing}
