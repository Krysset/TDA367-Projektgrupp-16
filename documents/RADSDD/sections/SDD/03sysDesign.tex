\section{System design}
\begin{comment}
	Draw an UML package diagram for the top level for all components that you have
	identified above (which can be just one if you develop a standalone application).
	Describe the interfaces and dependencies between the packages. Describe how you
	have implemented the MVC design pattern.

	Create an UML class diagram for every package. One of the packages will contain
	the model of your application. This will be the design model of your
	application, describe in detail the relation between your domain model and your
	design model. There should be a clear and logical relation between the two. Make
	sure that these models stay in `sync' during the development of your application.

	Describe which (if any) design patterns you have used.

	The above describes the static design of your application. It may sometimes be
	necessary to describe the \emph{dynamic} design of your application as well. You
	can use an UML \emph{sequence diagram} to show the different parts of your
	application communicate an in what order.
\end{comment}
\subsection{Packages}
\begin{itemize}
	\item controller
	      \begin{itemize}
		      \item controller.combat
		            \subitem controller.combat.ui
		      \item controller.enums
	      \end{itemize}
	\item interfaces
	\item model
	      \begin{itemize}
		      \item model.combat
		            \subitem model.combat.actions
		            \subitem model.combat.actions.abilities
		            \subitem model.combat.creatures
		      \item model.creature
		      \item model.overworld
		            \subitem model.overworld.encounter
		            \subitem model.overworld.map
		      \item model.player
	      \end{itemize}
	\item Util
	\item view
	      \begin{itemize}
		      \item view.combat
		      \item view.components
		      \item view.overworld
		            \subitem view.overworld.textureMap
		      \item vew.player
		      \item view.scenes
		      \item view.utils
	      \end{itemize}
\end{itemize}

\subsection{Controller}
\label{controller}
\includesvg[scale=0.5]{images/com.g16.feyrune.controller.svg}
\subsection{Model}
\label{model}
\includesvg[scale=0.5]{images/com.g16.feyrune.model.svg}
\subsection{CombatModel}
\label{combatmodel}
\includesvg[scale=0.5]{images/com.g16.feyrune.CombatModel.svg}
\subsection{OverworldModel}
\label{overworldmodel}
\includesvg[scale=0.5]{images/com.g16.feyrune.OverworldModel.svg}
\subsection{View}
\label{view}
\includesvg[scale=0.5]{images/com.g16.feyrune.view.svg}
\subsection{CombatScene}
\label{combatscene}
\includesvg[scale=0.5]{images/com.g16.feyrune.view.combatScene.svg}
\subsection{OverworldScene}
\label{overworldscene}
\includesvg[scale=0.5]{images/com.g16.feyrune.view.overworldScene.svg}

\subsection{Design Patterns}
Throughout the code, Group 16 has tried to implement design patterns wherever feasible. Examples of design patterns used are:
\paragraph{Model-View-Controller\\}
\paragraph{Factory pattern\\}
\paragraph{State pattern (hopefully)\\}
\paragraph{Observer pattern\\}
\paragraph{Singleton\\}
\paragraph{Façade pattern\\}
