\section{Quality}
\begin{comment}
	\begin{itemize}
		\item Describe how you test your application and where to find these tests. If
			  applicable, give a link to your continuous integration.
		\item List all known issues.
		\item Run analytical tools on your software and show the results. Use for example:
			  \begin{itemize}
				  \item Dependencies: \href{http://stan4j.com/}{STAN} or similar.
				  \item Quality tool reports, like \href{http://filehippo.com/download_pmd/}{PMD}.
			  \end{itemize}
	\end{itemize}

	NOTE: Each Java, XML, etc. file should have a header comment: Author,
	responsibility, used by ..., uses ..., etc.
\end{comment}
\subsection{Application quality}
To ensure the application is of great quality, every update of it goes through several levels of testing, beginning with a personal review from the code writer. When the writer feels their contribution is good enough, the code is tested using JUnit\cite{JUnit} to make sure it functions logically correct, even in common corner cases.\\
\\
After the JUnit testing, a pull request is created, forcing at least one other contributor to analyze it and test the code themselves before it being added to the source code.\\
\\
All tests written can be found in the \href{https://github.com/Krysset/TDA367-Projektgrupp-16/tree/main/core/src/test/java/com/g16/feyrune}{"core/src/tests"-folder}.

\subsection{Issues}
Following is a list of all known issues:
\begin{itemize}[$\circ$]
	\item The different creatures aren't shown during combat.
	\item What ability to use in combat is chosen for you.
	\item Creatures regain full health after ever encounter.
\end{itemize}

\begin{comment}
\subsection{Analytics feedback}
%TODO: run stan4j and pmd on the project and write down the results here

\subsection{Access control and security}
This section shan't be needed
\end{comment}